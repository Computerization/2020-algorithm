\documentclass[UTF8,12pt]{ctexart}
\usepackage{../Writeup}
\title{斐波那契数列}
\author{Josh-Cena}
\begin{document}
\maketitle
\begin{mdframed}[style=Question]
    斐波那契数列:
    \[F_{n}=
    \begin{cases}
        0,&n=0\\
        1,&n=1\\
        F_{n-2}+F_{n-1},&n>1
    \end{cases}\]

    给定$n$,求$F_{n}\text{ mod }10^9+7$。

    \begin{table}[H]
        \centering
        \begin{tabular}{|c|c|c|}\hline
            数据规模&内存限制&运行时间\\\hline
            $0\le n\le 10^{19}$&\SI{64}{MB}&\SI{1.0}{s}\\\hline
        \end{tabular}
    \end{table}
\end{mdframed}
\textit{题解.} $10^{19}$显然灭掉了所有用循环解决的想法。有没有比简单的$\mathcal{O}(n)$更好一点的方法?用\textbf{矩阵快速幂},可以达到$\mathcal{O}(\log n)$。观察到:
\[\begin{pmatrix}F_{n+1}\\F_{n+2}\end{pmatrix}=\begin{pmatrix}F_{n+1}\\F_{n}+F_{n+1}\end{pmatrix}=\begin{pmatrix}0&1\\1&1\end{pmatrix}\begin{pmatrix}F_{n}\\F_{n+1}\end{pmatrix}\]

这一步对于所有递推数列都是适用的,因此在有经验之后应该非常容易得到。一般地,对于$F_{n+2}=aF_{n}+bF_{n+1}$,有
\[\begin{pmatrix}F_{n+1}\\F_{n+2}\end{pmatrix}=\begin{pmatrix}F_{n+1}\\aF_{n}+bF_{n+1}\end{pmatrix}=\begin{pmatrix}0&1\\a&b\end{pmatrix}\begin{pmatrix}F_{n}\\F_{n+1}\end{pmatrix}\]

从递推式中有
\[\begin{pmatrix}F_{n+m}\\F_{n+m+1}\end{pmatrix}=\begin{pmatrix}0&1\\1&1\end{pmatrix}^m\begin{pmatrix}F_{n}\\F_{n+1}\end{pmatrix}\]

取$n=0$,得到
\[\begin{pmatrix}F_{m}\\F_{m+1}\end{pmatrix}=\begin{pmatrix}0&1\\1&1\end{pmatrix}^m\begin{pmatrix}F_{0}\\F_{1}\end{pmatrix}\]

因此把问题转化成了如何求矩阵$m$次方的问题。如果设$m=2^0a_0+2^1a_1+2^2a_2+\dots$(也就是把$m$用二进制表示),那么有
\[\begin{pmatrix}0&1\\1&1\end{pmatrix}^m=\left(\begin{pmatrix}0&1\\1&1\end{pmatrix}^{1}\right)^{a_0}\times \left(\begin{pmatrix}0&1\\1&1\end{pmatrix}^{2}\right)^{a_1}\times \left(\begin{pmatrix}0&1\\1&1\end{pmatrix}^{4}\right)^{a_2}\dots\]

而这些矩阵的$2^k$次方,完全可以预处理。当$m$的数量级为$10^{19}$时,$k<\log_2 10^{19}<64$,最多只需要存储63个矩阵。并且
\[\begin{pmatrix}0&1\\1&1\end{pmatrix}^{2^k}=\begin{pmatrix}0&1\\1&1\end{pmatrix}^{2^{k-1}}\times \begin{pmatrix}0&1\\1&1\end{pmatrix}^{2^{k-1}}\]

这些乘方可以在$\mathcal{O}(\log m)$时间内得到。这便是快速幂的思想:计算所有的$2^k$次方,然后把其中需要的那些组合起来即可。

下面是 C++ 代码,其中最繁琐的部分是实现矩阵乘法:

\begin{lstlisting}
#include <iostream>
#include <cmath>

using namespace std;

unsigned long long mat_pow[64][4] = {
    {0,1,1,1}
};

int fib(unsigned long long k){
    unsigned long long t00 = 0, t01 = 1, t10 = 1, t11 = 1;
    for (int i = 0; i < 64; i++) {
        if (k & (1ull << i)) {
            unsigned long long a = (t00 * mat_pow[i][0] + t01 * mat_pow[i][2]) % 1000000007;
            unsigned long long b = (t00 * mat_pow[i][1] + t01 * mat_pow[i][3]) % 1000000007;
            unsigned long long c = (t10 * mat_pow[i][0] + t11 * mat_pow[i][2]) % 1000000007;
            unsigned long long d = (t10 * mat_pow[i][1] + t11 * mat_pow[i][3]) % 1000000007;
            t00 = a;
            t01 = b;
            t10 = c;
            t11 = d;
        }
    }
    return t00 % 1000000007;
}

int main(){
    for (int i = 1; i < 64; i++) {
        mat_pow[i][0] = (mat_pow[i-1][0] * mat_pow[i-1][0] + mat_pow[i-1][1] * mat_pow[i-1][2]) % 1000000007;
        mat_pow[i][1] = (mat_pow[i-1][0] * mat_pow[i-1][1] + mat_pow[i-1][1] * mat_pow[i-1][3]) % 1000000007;
        mat_pow[i][2] = (mat_pow[i-1][0] * mat_pow[i-1][2] + mat_pow[i-1][2] * mat_pow[i-1][3]) % 1000000007;
        mat_pow[i][3] = (mat_pow[i-1][1] * mat_pow[i-1][2] + mat_pow[i-1][3] * mat_pow[i-1][3]) % 1000000007;
    }
    unsigned long long n;
    cin >> n;
    cout << fib(n) << endl;
    return 0;
}    
\end{lstlisting}
\end{document}
