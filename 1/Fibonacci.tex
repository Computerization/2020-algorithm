\documentclass[UTF8,12pt]{ctexart}
\usepackage{../Writeup}
\title{斐波那契数列}
\author{Josh-Cena}
\begin{document}
\maketitle
\begin{mdframed}[style=Question]
    斐波那契数列:
    \[F_{n}=
    \begin{cases}
        0,&n=0\\
        1,&n=1\\
        F_{n-2}+F_{n-1},&n>1
    \end{cases}\]

    给定$n$,求$F_{n}\text{ mod }10^9+7$。

    \begin{table}[H]
        \centering
        \begin{tabular}{|c|c|c|}\hline
            数据规模&内存限制&运行时间\\\hline
            $0\le n\le 10^{19}$&\SI{64}{MB}&\SI{1.0}{s}\\\hline
        \end{tabular}
    \end{table}
\end{mdframed}
\textit{题解.} $10^{19}$显然灭掉了所有用循环解决的想法。有没有比简单的$\mathcal{O}(n)$更好一点的方法?用\textbf{矩阵快速幂},可以达到$\mathcal{O}(\log n)$。观察到:
\[\begin{pmatrix}F_{n+1}\\F_{n+2}\end{pmatrix}=\begin{pmatrix}F_{n+1}\\F_{n}+F_{n+1}\end{pmatrix}=\begin{pmatrix}0&1\\1&1\end{pmatrix}\begin{pmatrix}F_{n}\\F_{n+1}\end{pmatrix}\]

这一步对于所有递推数列都是适用的,因此在有经验之后应该非常容易得到。一般地,对于$F_{n+2}=aF_{n}+bF_{n+1}$,有
\[\begin{pmatrix}F_{n+1}\\F_{n+2}\end{pmatrix}=\begin{pmatrix}F_{n+1}\\aF_{n}+bF_{n+1}\end{pmatrix}=\begin{pmatrix}0&1\\a&b\end{pmatrix}\begin{pmatrix}F_{n}\\F_{n+1}\end{pmatrix}\]

从递推式中有
\[\begin{pmatrix}F_{n+m}\\F_{n+m+1}\end{pmatrix}=\begin{pmatrix}0&1\\1&1\end{pmatrix}^m\begin{pmatrix}F_{n}\\F_{n+1}\end{pmatrix}\]

取$n=0$,得到
\[\begin{pmatrix}F_{m}\\F_{m+1}\end{pmatrix}=\begin{pmatrix}0&1\\1&1\end{pmatrix}^m\begin{pmatrix}F_0\\F_1\end{pmatrix}\]

因此把问题转化成了如何求矩阵$m$次方的问题。如果设$m=2^0a_0+2^1a_1+2^2a_2+\dots$(也就是把$m$用二进制表示),那么有
\[\begin{pmatrix}0&1\\1&1\end{pmatrix}^m=\left(\begin{pmatrix}0&1\\1&1\end{pmatrix}^{1}\right)^{a_0}\times \left(\begin{pmatrix}0&1\\1&1\end{pmatrix}^{2}\right)^{a_1}\times \left(\begin{pmatrix}0&1\\1&1\end{pmatrix}^{4}\right)^{a_2}\dots\]

而这些矩阵的$2^k$次方,完全可以预处理。当$m$的数量级为$10^{19}$时,$k<\log_2 10^{19}<64$,最多只需要存储63个矩阵。并且
\[\begin{pmatrix}0&1\\1&1\end{pmatrix}^{2^k}=\begin{pmatrix}0&1\\1&1\end{pmatrix}^{2^{k-1}}\times \begin{pmatrix}0&1\\1&1\end{pmatrix}^{2^{k-1}}\]

这些乘方可以在$\mathcal{O}(\log m)$时间内得到。这便是快速幂的思想:计算所有的$2^k$次方,然后把其中需要的那些组合起来即可。

下面是 C++ 代码,其中最繁琐的部分是实现矩阵乘法:

\lstinputlisting{./Fibonacci.cpp}

补充一下矩阵的乘法公式:
\[\begin{pmatrix}a_0&a_1\\a_2&a_3\end{pmatrix}\times\begin{pmatrix}b_0&b_1\\b_2&b_3\end{pmatrix}=\begin{pmatrix}a_0b_0+a_1b_2&a_0b_1+a_1b_3\\a_2b_0+a_3b_2&a_2b_1+a_3b_3\end{pmatrix}\]
\[\begin{pmatrix}a_0&a_1\\a_2&a_3\end{pmatrix}\times\begin{pmatrix}b_0\\b_1\end{pmatrix}=\begin{pmatrix}a_0b_0+a_1b_1\\a_2b_0+a_3b_1\end{pmatrix}\]
\end{document}
